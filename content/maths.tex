\section{Maths}
\todo{infinite/finite sums, FFT}
\todo{permutation to/from index}
\todo{Game Theory, Grundy Theorem}


\subsection{Modular multiplication}
\bigo{logN} Time

Computes $(a * b)$\:\%\:mod without overflow.
\lstinputlisting{source/mod_mul.cpp}


\subsection{Binary Exponentiation}
\bigo{logN} Time

Returns ${base^{exp}}$\:\%\:mod

Works with \verb|int|
\lstinputlisting{source/bin_expo.cpp}

Works with \verb|long long| \\
Requires \verb|mod_mul(ll, ll, ll)|
\lstinputlisting{source/bin_expoll.cpp}


\subsection{Modulo Inverse}
\bigo{logN} Time

Returns ${x^{-1}}$\:\%\:mod

Works with \verb|int| \\
Requires \verb|bin_expo(int, int, int)|
\lstinputlisting{source/mod_inverse.cpp}


Works with \verb|long long| \\
Requires \verb|bin_expoll(ll, ll, ll)|
\lstinputlisting{source/mod_inversell.cpp}


\subsection{Factorial}
\bigo{N} Time

Initializes \verb|factorial| and \verb|inv_factorial| \\
If \verb|mod| $> 2^{32}$, \verb|mod_mul(ll, ll, ll)| must be used instead of \verb|*| \\
and \verb|mod_inversell(ll, ll)| must be used instead of \verb|mod_inverse(int, int)|

Requires \verb|mod_inverse(int, int)|
\lstinputlisting{source/factorial.cpp}


\subsection{Binomial Coefficient}
\bigo{1} Time

Computes $\binom{n}{k}$\:\%\:mod \\
If \verb|mod| $> 2^{32}$, \verb|mod_mul(ll, ll, ll)| must be used instead of \verb|*|

\verb|init_factorials(int, int)| must be called before this function!
\lstinputlisting{source/binomial_coefficient.cpp}

\todo{Add list of problems solved by Catalan numbers}

\subsection{Catalan Number}
\bigo{1} Time

Computes the n-th Catalan number \\
If \verb|mod| $> 2^{32}$, \verb|mod_mul(ll, ll, ll)| must be used instead of \verb|*|

\verb|init_factorials(int, int)| must be called before this function!
\lstinputlisting{source/catalan_number.cpp}


\subsection{Prime sieve}
Calculates \verb|isprime| for all $x \in [0 \ldots N]$, \bigo{NloglogN}.
\lstinputlisting{source/erathostenes.cpp}

\subsection{(Deterministic) Miller-Rabin - Primality Check}
Returns true if the number is prime, false otherwise

Requires \verb|mod_mul(ll, ll, ll), mod_expoll(ll, ll, ll)|
\lstinputlisting{source/miller_rabin.cpp}


\subsection{Pollard's Rho - Prime Factor}
Returns n if n is prime or n is $< 2$ \\
Else returns a prime factor of n

Requires \verb|miller_rabin(ll), mod_mul(ll, ll, ll)|
\lstinputlisting{source/pollard_rho.cpp}


\subsection{Integer Factorization}
Returns a sorted list of prime factors of n \\
Each factor is represented as a \verb|pair<ll, int>| \\
where the first element is the prime factor and the second is the power of that factor

Requires \verb|pollard_rho(ll)|
\lstinputlisting{source/factorize.cpp}


\subsection{Extended Euclidean Algorithm}
\bigo{log(min(a, b))} Time

Solves $ax + by = gcd(a, b)$ \\
\verb|auto [x, y, g] = egcd(l, n[i]);|
\lstinputlisting{source/egcd.cpp}


\subsection{Diophantine Linear Equations}
\bigo{log(min(a, b))} Time

Solves $ax + by = c$ \\
Returns \verb|true| if there's a solution, \verb|false| otherwise \\
\verb|diophantine(a, b, c, x0, y0, g);| \\
If there's a solution the function will do side-effect on \verb|x0, y0, g|\\
All the possible solutions have the form: \\
\verb|x = x0 + (b / g) * k| \\
\verb|y = y0 - (a / g) * k| \\
with $k \in Z$

Requires \verb|egcd(ll, ll)|
\lstinputlisting{source/diophantine.cpp}


\subsection{Chinese Remainder Theorem}
Returns x in range [0, $\prod{n[i]}$) s.t.
$x = a[i] \: mod \: n[i], \forall i$ \\
It there's no solution, it returns -1

Requires \verb|egcd(ll, ll)|
\lstinputlisting{source/chinese_remainder_theorem.cpp}


\subsection{Matrix multiplication \& power}
Functions for matrix-matrix multiplication, matrix-vector multiplication and matrix power. All elements are calculated modulo \verb|mod|. If \verb|mod| $> 2^{32}$, \verb|mod_mul| must be used instead of \verb|*| for multiplication inside of \verb|mat_mul|
\lstinputlisting{source/matrix_mult_pow.cpp}

\subsection{Linear recurrences}
Solve for the n-th term of a linear recurrence in \bigo{(k+h)^3logn}.\\
$x_i = a_0 x_{i-1} + ... + a_{k-1} x_{i-k} \;\; + \;\; b_0 + b_1 i + ... + b_{h-1} i^h$ \\
Given $x_0, x_1, ..., x_{k-1}$ calculates $x_n \% mod$.

Fibonacci: \texttt{linear\_recurrence(\{1,1\},\{\},\{0,1\},n,mod);}\\
Requires \verb|mat_mul|, \verb|mat_pow|
\lstinputlisting{source/linear_recurrences.cpp}


\subsection{Binomial Distribution}
Discrete probability distribution of the number of successes in a sequence
of $n$ indipendent Bernoulli trials, each with a probability $p$ of success.

\[P(X = k) = \binom{n}{k}p^k(1 - p)^{n - k}\]
\[E(X) = np,\,Var(X) = np(1 - p)\]


\subsection{Negative Binomial Distribution}
Discrete probability distribution of the number of successes in a sequence
of $n$ indipendent trials, each with a probability $p$ of success,
before $r$ failures.

\[P(X = k) = \binom{k + r - 1}{r - 1}(1 - p)^kp^r\]
\[E(X) = \frac{rp}{(1 - p)^2},\,Var(X) = np(1 - p)\]


\subsection{Geometric Distribution}
Discrete probability distribution of the number of Bernoulli trials,
each with a probability $p$ of success, needed to get one success.

\[P(X = k) = (1 - p)^kp\]
\[E(X) = \frac{1}{p},\,Var(X) = \frac{1 - p}{p^2}\]


\subsection{Poisson Distribution}
Discrete probability distribution of the number of events occurring
in a fixed interval $t$ if these events occur with a mean rate of $\lambda / t$
and indipendently of the time since the last event.

\[P(X = k) = e^{-\lambda}\frac{\lambda^k}{k!}\]
\[E(X) = \lambda,\,Var(X) = \lambda\]


\subsection{(Continuous) Uniform Distribution}
Continuous probability distribution that describes an experiment
where there is an arbitrary outcome that lies between $a$ and $b$.

\[f(x) =
	\begin{cases}
		\frac{1}{b-a} & a < x < b \\
		0 & \textrm{otherwise}
	\end{cases}
\]
\[E(X) = \frac{a + b}{2},\,Var(X) = \frac{(b - a)^2}{12}\]


\subsection{Normal Distribution}
Continuous probability distribution used to model phenomena that have
a default behaviour and cumulative possible deviations from that behaviour.

\[f(x) = \frac{1}{\sqrt{2\pi\sigma^2}}e^{-\frac{(x - \mu)^2}{2\sigma^2}}\]
\[E(X) = \mu,\,Var(X) = \sigma^2\]


\subsection{Exponential Distribution}
Continuous probability distribution of the times between events
in a Poisson process of parameter $\lambda$.
It's the continuous analogue of the geometric distribution and
it has the property of being memoryless.

\[f(x) =
	\begin{cases}
		\lambda e^{-\lambda x} & x \ge 0 \\
		0 & \textrm{otherwise}
	\end{cases}
\]
\[E(X) = \frac{1}{\lambda},\,Var(X) = \frac{1}{\lambda^2}\]
