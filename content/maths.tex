\section{Maths}
\todo{permutation to/from index}


\subsection{Modular multiplication}
\bigo{logN} Time

Computes $(a * b)$\:\%\:mod without overflow.
\lstinputlisting{source/mod_mul.cpp}


\subsection{Binary Exponentiation}
\bigo{logN} Time

Returns ${base^{exp}}$\:\%\:mod

Works with \verb|int|
\lstinputlisting{source/bin_expo.cpp}

Works with \verb|long long| \\
Requires \verb|mod_mul(ll, ll, ll)|
\lstinputlisting{source/bin_expoll.cpp}


\subsection{Modular Inverse}
\bigo{logN} Time

Returns ${x^{-1}}$\:\%\:mod. For prime moduli, the answer always exists. For non prime moduli, returns $-1$ if no answer exists, i.e. if \verb|gcd(n,mod)|$ \neq 1$.

Requires \verb|egcd(ll, ll)|
\lstinputlisting{source/mod_inverse.cpp}


\subsection{Factorial}
\bigo{N} Time

Initializes \verb|factorial| and \verb|inv_factorial| \\
If \verb|mod| $> 2^{32}$, use \verb|mod_mul(ll, ll, ll)| instead of \verb|*|

NOTE: This function works iff the modulo is coprime with $n!$
(for example $mod$ is a prime number strictly greater than $n$)

Requires \verb|mod_inverse(int, int)|
\lstinputlisting{source/factorial.cpp}


\subsection{Binomial Coefficient}
\bigo{1} Time

Computes $\binom{n}{k}$\:\%\:mod \\
If \verb|mod| $> 2^{32}$, use \verb|mod_mul(ll, ll, ll)| instead of \verb|*| \\
\verb|init_factorials(int, int)| must be called before this function!
\lstinputlisting{source/binomial_coefficient.cpp}


\subsection{Lucas's Theorem}
Let $m, n$ be non-negative integers and $p$ a prime number s.t.: \\
\[m = m_k p^k + m_{k-1} p^{k-1} + \ldots + m_1 p + m_0\]
\[n = n_k p^k + n_{k-1} p^{k-1} + \ldots + n_1 p + n_0\]
The following congruence relation holds:
\[\binom{m}{n} \equiv \prod_{i=0}^k\binom{m_i}{n_i} \pmod{p}\]


\subsection{Catalan Number}
\bigo{1} Time

N-th Catalan number, $C_n = \frac{1}{n+1}\binom{2n}{n} = \frac{4n-2}{n+1}C_{n-1}$

Problems solved by Catalan Numbers:
\begin{itemize}
	\item Number of correct bracket sequence consisting of $n$ opening and $n$ closing brackets
	\item Number of rooted full binary tree with $n + 1$ leaves
	\item Number of triangulations of a convex polygon with $n + 2$ sides
	\item Number of ways to connect $2n$ points on a circle to form $n$ disjoint chords
	\item Number of monotonic lattice paths from point $(0, 0)$ to point $(n, n)$ in a square of $n \times n$, which do not pass above the main diagonal
\end{itemize}


\subsection{Burnside's lemma}
Let $G$ be a finite group that acts on a set $X$. \\
For each $g$ in $G$ let $X^g$ denote the set of elements in $X$ invariant to $g$. \\
The number of orbits $|X / G|$ is:
\[|X / G| = \frac{1}{|G|} \sum_{g \in G}|X^g|\]

It is easier to apply than it seems, here's an example: \\
Count the number of rotationally distinct colourings of a cube using $n$ different colours.

Two elements belong to the same orbit if one is the rotation of the other,
so we have to count the number of orbits. \\
There are 24 elements of $G$:

\begin{enumerate}
	\item The identity element: $n^6$ elements are invariant to it
	\item 6 90-degree rotations: each has $n^3$ elements invariant to it
	\item 3 180-degree rotations: each has $n^4$ elements invariant to it
	\item 8 120-degree rotations: each has $n^2$ elements invariant to it
	\item 6 180-degree rotations: each has $n^3$ elements invariant to it
\end{enumerate}

So, in total we have:
\[|X / G| = \frac{n^6 + 6n^3 + 3n^4 + 8n^2 + 6n^3}{24} = \]
\[= \frac{n^6 + 3n^4 + 12n^3 + 8n^2}{24}\]


If \verb|mod| $> 2^{32}$, use \verb|mod_mul(ll, ll, ll)| instead of \verb|*| \\
\verb|init_factorials(int, int)| must be called before this function!
\lstinputlisting{source/catalan_number.cpp}


\subsection{Prime sieve}
Calculates \verb|isprime| for all $x \in [0 \ldots N]$, \bigo{NloglogN}.
\lstinputlisting{source/erathostenes.cpp}

\subsection{(Deterministic) Miller-Rabin - Primality Check}
Returns true if the number is prime, false otherwise

Requires \verb|mod_mul(ll, ll, ll), mod_expoll(ll, ll, ll)|
\lstinputlisting{source/miller_rabin.cpp}


\subsection{Pollard's Rho - Prime Factor}
Returns n if n is prime or n is $< 2$ \\
Else returns a prime factor of n

Requires \verb|miller_rabin(ll), mod_mul(ll, ll, ll)|
\lstinputlisting{source/pollard_rho.cpp}


\subsection{Integer Factorization}
Returns a sorted list of prime factors of n \\
Each factor is represented as a \verb|pair<ll, int>| \\
where the first element is the prime factor and the second is the power of that factor

Requires \verb|pollard_rho(ll)|
\lstinputlisting{source/factorize.cpp}


\subsection{Extended Euclidean Algorithm}
\bigo{log(min(a, b))} Time

Solves $ax + by = gcd(a, b)$ \\
\verb|auto [x, y, g] = egcd(l, n[i]);|
\lstinputlisting{source/egcd.cpp}


\subsection{Diophantine Linear Equations}
\bigo{log(min(a, b))} Time

Solves $ax + by = c$ \\
Returns \verb|true| if there's a solution, \verb|false| otherwise \\
\verb|diophantine(a, b, c, x0, y0, g);| \\
If there's a solution the function will do side-effect on \verb|x0, y0, g|\\
All the possible solutions have the form: \\
\verb|x = x0 + (b / g) * k| \\
\verb|y = y0 - (a / g) * k| \\
with $k \in Z$

Requires \verb|egcd(ll, ll)|
\lstinputlisting{source/diophantine.cpp}


\subsection{Chinese Remainder Theorem}
Returns x in range [0, $\prod{n[i]}$) s.t.
$x = a[i] \: mod \: n[i], \forall i$ \\
It there's no solution, it returns -1

Requires \verb|egcd(ll, ll)|
\lstinputlisting{source/chinese_remainder_theorem.cpp}


\subsection{Euler Phi function}
$\phi(n) = $ number of integers in $[1\ldots n]$ coprime with n. If $n$ is prime, $\phi(n)=n-1$.

Useful related lemma: \\
If $d$ is a divisor of $n$, then there are $\phi(n / d)$ numbers $i \leq n$ s.t. $gcd(i, n) = d$

\verb|phi| returns $\phi(n)$ in \bigo{\sqrt n\log n} \\
\verb|phi_up_to_n| returns $\phi(i) \; \forall i \in [0 \ldots n]$ in \bigo{n \log\log n}

\lstinputlisting{source/phi.cpp}


\subsection{Fast Fourier Transform}
\bigo{(n + m) \log (n + m)}

Given the \verb|a| and \verb|b| the coefficients of two polynomials: \\
$A(x) = a_0 x^0 * \ldots * a_{n-1} x^{n-1}$, $B(x) = b_0 x^0 * \ldots * b_{m-1} x^{m-1}$ \\
Let $C(x) = A(x) * B(x)$

\verb|conv(a, b)| computes the coefficients of the polynomial $C(x)$ \\
Works with \verb|long double|

\verb|mod_conv(a, b, mod)| computes the coefficients modulo \verb|mod| \\
Works with \verb|long long|

\lstinputlisting{source/fft.cpp}


\subsection{Probability Distributions}

\begin{itemize}[leftmargin=3ex]

\item \textbf{Binomial Distribution}
Discrete probability distribution of the number of successes in a sequence
of $n$ indipendent Bernoulli trials, each with a probability $p$ of success.
\[P(X = k) = \binom{n}{k}p^k(1 - p)^{n - k}\]
\[E(X) = np,\,Var(X) = np(1 - p)\]


\item \textbf{Negative Binomial Distribution}
Discrete probability distribution of the number of successes in a sequence
of $n$ indipendent trials, each with a probability $p$ of success,
before $r$ failures.
\[P(X = k) = \binom{k + r - 1}{r - 1}(1 - p)^kp^r\]
\[E(X) = \frac{rp}{(1 - p)^2},\,Var(X) = np(1 - p)\]


\item \textbf{Geometric Distribution}
Discrete probability distribution of the number of Bernoulli trials,
each with a probability $p$ of success, needed to get one success.
\[P(X = k) = (1 - p)^kp\]
\[E(X) = \frac{1}{p},\,Var(X) = \frac{1 - p}{p^2}\]


\item \textbf{Poisson Distribution}
Discrete probability distribution of the number of events occurring
in a fixed interval $t$ if these events occur with a mean rate of $\lambda / t$
and indipendently of the time since the last event.
\[P(X = k) = e^{-\lambda}\frac{\lambda^k}{k!}\]
\[E(X) = \lambda,\,Var(X) = \lambda\]


\item \textbf{(Continuous) Uniform Distribution}
Continuous probability distribution that describes an experiment
where there is an arbitrary outcome that lies between $a$ and $b$.
\[f(x) =
	\begin{cases}
		\frac{1}{b-a} & a < x < b \\
		0 & \textrm{otherwise}
	\end{cases}
\]
\[E(X) = \frac{a + b}{2},\,Var(X) = \frac{(b - a)^2}{12}\]


\item \textbf{Normal Distribution}
Continuous probability distribution used to model phenomena that have
a default behaviour and cumulative possible deviations from that behaviour.
\[f(x) = \frac{1}{\sqrt{2\pi\sigma^2}}e^{-\frac{(x - \mu)^2}{2\sigma^2}}\]
\[E(X) = \mu,\,Var(X) = \sigma^2\]


\item \textbf{Exponential Distribution}
Continuous probability distribution of the times between events
in a Poisson process of parameter $\lambda$.
It's the continuous analogue of the geometric distribution and
it has the property of being memoryless.
\[f(x) =
	\begin{cases}
		\lambda e^{-\lambda x} & x \ge 0 \\
		0 & \textrm{otherwise}
	\end{cases}
\]
\[E(X) = \frac{1}{\lambda},\,Var(X) = \frac{1}{\lambda^2}\]

\end{itemize}

\subsection{Sums}
\[\sum_{i=1}^{n}{i} = \frac{n(n+1)}{2},\; \sum_{i=1}^{n}{i^2} = \frac{n(n+1)(2n+1)}{6},\; \sum_{i=1}^{n}{i^3} = \frac{n^2(n+1)^2}{4}\]
\[\sum_{i=0}^{n}{c^i} = \frac{c^{n+1}-1}{c-1}, c \neq 1,\; \sum_{i=0}^{\infty}{c^i} = \frac{1}{1-c}, |c| < 1\]
\[\sum_{i=0}^{n}{ic^i} = \frac{nc^{n+2}-(n+1)c^{n+1}+c}{(c-1)^2}, c \neq 1,\; \sum_{i=0}^{\infty}{ic^i} = \frac{c}{(1-c)^2}, |c| < 1\]

\[\sum_{k=0}^{n}{\binom{n}{k}} = 2^n,\; \binom{n}{m}\binom{m}{k} = \binom{n}{k}\binom{n-k}{m-k}\]
\[\sum_{k=0}^{n}{\binom{r+k}{k}} = \binom{r+n+1}{n},\; \sum_{k=m}^{n}{\binom{k}{m}} = \binom{n+1}{m+1}\]
\[\sum_{k=0}^{n}{\binom{r}{k}\binom{s}{n-k} = \binom{r+s}{n}}\]


\subsection{Möbius Function}
$\mu(n)$ is a multiplicative function s.t.: \\
\[\mu(n) =
	\begin{cases}
		1 & n = 1 \\
		0 & \textrm{$n$ has a squared prime factor} \\
		(-1)^k & \textrm{$n$ is the product of $k$ different primes}
	\end{cases}
\]

\verb|mu_up_to_n| returns $\mu(i) \; \forall i \in [0, n]$ in \bigo{n \log\log n}
\lstinputlisting{source/mobius_function.cpp}


\subsection{Möbius Inversion}
Notation used:
\[[<expr>] =
	\begin{cases}
		1 & <expr> = true \\
		0 & otherwise
	\end{cases}
\]

Möbius Inversion definition:
\[g(n) = \sum_{d|n}f(d) <=> f(n) = \sum_{d|n}g(d)\mu(\frac{n}{d})\]

Property:
\[\sum_{d|n}\mu(d) = [n = 1]\]

Example where this property could be used:
Number of coprime integers $x, y \in [1, n]$
\[f(n) = \sum_{i=1}^{n}\sum_{j=1}^{n}[gcd(i,j) = 1] =\]
\[= \sum_{i=1}^{n}\sum_{j=1}^{n}\sum_{d|gcd(i,j)}\mu(d) =\]
\[= \sum_{i=1}^{n}\sum_{j=1}^{n}\sum_{d=1}^{n}\mu(d)[d | gcd(i, j)] =\]
\[= \sum_{i=1}^{n}\sum_{j=1}^{n}\sum_{d=1}^{n}\mu(d)[d | i][d | j] =\]
\[= \sum_{d=1}^{n}\mu(d)\sum_{i=1}^{n}[d | i]\sum_{j=1}^{n}[d | j] =\]
\[= \sum_{d=1}^{n}\mu(d)\floor{\frac{n}{d}}^2\]


\subsection{XOR of every number $x \in [1, n]$}
\[1 \oplus \ldots \oplus n =
	\begin{cases}
		n & n \% 4 = 0 \\
		1 & n \% 4 = 1 \\
		n + 1 & n \% 4 = 2 \\
		0 & n \% 4 = 3
	\end{cases}
\]


\subsection{Sprague-Grundy's Theorem}
Mex = minimum non-negative integer not contained in a set

Let's consider a two-player impartial game. \\
The game is equivalent (if played optimally) to a Nim game with a single
heap of size equal to the nimber (or Grundy value) of the state. \\
The nimber $g$ of a state $x$ is defined as:

$g(x)$ = $mex$(\{$g(y)$ for $y$ reachable from $x$ in one move\})

From this formula it is obvious that the nimber of a losing state is always 0.

In practice we could have multiple games simultaneously that share the same
dp table of nimbers, every game is equivalent to a different heap in a Nim game.

To solve the Nim game it is sufficient to take the XOR of every heap size.
Player one wins if the result is different from zero, loses otherwise.
