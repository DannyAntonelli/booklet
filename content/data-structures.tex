\section{Data structures}

\subsection{Ordered set/map}
\snippet{source/pbds-set.h}

\subsection{Faster hashmap}
3x faster, uses 1.5x memory
\snippet{source/pbds-hashmap.h}

\subsection{Fenwick tree}
\bigo{N} build, \bigo{logN} point update and range query.
\snippet{source/fenwick.h}

\subsection{2D Fenwick}
\bigo{log^2 N} point update and range query.
To extend to more dimensions, add more nested loops in update
and query, and complete the last query by inclusion-exclusion.
\snippet{source/fenwick2d.h}

\subsection{Segment tree}
\bigo{N} construction, \bigo{logN} per operation.
\snippet{source/segment.h}

\subsection{Segment tree - lazy range updates}
\bigo{N} construction, \bigo{logN} per operation.
\snippet{source/segment-lazy.h}

\subsection{Segment tree - persistent}
\bigo{N} construction, \bigo{logN} per operation.
Creates \bigo{logN} new nodes per update.
\snippet{source/segment-persistent.h}

\subsection{Segment tree - persistent with lazy range updates}
\bigo{N} construction, \bigo{logN} per operation.
Creates \bigo{logN} new nodes per update and query.
\snippet{source/segment-lazystent.h}

\subsection{Segment tree - sparse}
\bigo{logN} per operation, \bigo{QlogN} memory (for Q = queries + updates).\\
Works like a \verb|LazyST| but only creates nodes reached in queries and updates.\\
This allows us to work with large ranges in [\verb|0|, \verb|inf|] (where \verb|inf| can go up to \verb|LLONG_MAX/2|).\\
There is no constructor as the struct starts empty and must be initialized with multiple \verb|update|s.\\
Last tip: high \verb|inf| might cause overheads, so try to keep it low.
\snippet{source/segment-sparse.h}

\subsection{Union find}
\snippet{source/ufds.h}

\subsection{Sparse table}
\bigo{NlogN} construction and memory usage, \bigo{1} query.
\snippet{source/sparse-table.h}

\subsection{Minqueue}
Keeps the minimum over a sliding window, allows for inserting elements to the right and for removing all elements from the left until a certain index. All operations amortized \bigo{1}.
\snippet{source/minqueue.h}

\subsection{Mo queries}
Offline queries on a range in \bigo{(N+Q)\sqrt{N}}. Complete the add, remove and getresult functions.
\snippet{source/mo_queries.h}

\subsection{LineContainer - Convex Hull Trick}

\bigo{logN} time per operation \\
\verb|LineContainer cht;| \\
\verb|cht.add(k, m);| to add a line of the form $y = kx + m$ \\
\verb|cht.query(x);| to query the maximum value with coordinate $x$
\snippet{source/line_container.h}

\subsection{Treap}

Stores a sequence of values and allows access by index. Supports \texttt{split} and \texttt{merge} operations, and generic queries and lazy updates on ranges. Subsequence reversal and range sums are implemented as an example, change \texttt{Treap::recalc} and \texttt{Treap::push} (and fields in \texttt{Treap} if needed) to implement new operations.
The operations modify the tree, so always reassign the tree to the return value of operations: \texttt{t = insert(t, i, val)}. To perform a range query/update, use \texttt{range\_operation}.
All operations \bigo{logN}.

Usage:
\begin{lstlisting}
Treap *t = NULL;
t = insert(t, i, s[i]);

int value;
t = erase(t, pos, &value); // third argument can also be NULL

auto [l,r] = split(t, pos); // don't use t after this

t = range_operation(t, a, b, [](Treap *u) {
	u->rev ^= 1;
});

visit(t, [](Treap *u) {
	cout << u->val;
});
\end{lstlisting}
\snippet{source/treap.h}


\subsection{Link Cut Tree}
\bigo{\log N} per operation
\snippet{source/link_cut_tree.h}
