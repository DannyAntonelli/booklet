\section{Data structures}
\todo{Sparse Segment Tree}

\subsection{Ordered set/map}
\lstinputlisting{source/pbds-set.cpp}

\subsection{Faster hashmap}
3x faster, uses 1.5x memory
\lstinputlisting{source/pbds-hashmap.cpp}

\subsection{Fenwick tree}
\bigo{N} build, \bigo{logN} point update and range query.
\lstinputlisting{source/fenwick.cpp}

\subsection{2D Fenwick}
\bigo{log^2 N} point update and range query.
To extend to more dimensions, add more nested loops in update
and query, and complete the last query by inclusion-exclusion.
\lstinputlisting{source/fenwick2d.cpp}

\subsection{Lower bound on Fenwick tree}
If there are no negative values finds the
smallest i such that sum(0, i)>=k in \bigo{logN}.
If such i doesn't exists returns -1.
This function must be copied inside the struct
\lstinputlisting{source/fenwick_lower_bound.cpp}

\subsection{Segment tree}
\bigo{N} construction, \bigo{logN} per operation.
\lstinputlisting{source/segment.cpp}

\subsection{Segment tree - lazy range updates}
\bigo{N} construction, \bigo{logN} per operation.
\lstinputlisting{source/segment-lazy.cpp}

\subsection{Segment tree - persistent}
\bigo{N} construction, \bigo{logN} per operation.
Creates \bigo{logN} new nodes per update.
\lstinputlisting{source/segment-persistent.cpp}

\subsection{Lower bound on segment tree}
Finds the smallest i such that query(0, i).field>=k in \bigo{logN}.
If such i doesn't exists returns -1.
Only works in lazy and basic segment trees.
Only works if field is monotone. The field must be specified in cmp.
Both functions must be copied inside the segment struct.
\lstinputlisting{source/segment_lower_bound.cpp}


\subsection{Union find}
\lstinputlisting{source/ufds.cpp}

\subsection{Sparse table}
\bigo{NlogN} construction and memory usage, \bigo{1} query.
\lstinputlisting{source/sparse-table.cpp}

\subsection{Minqueue}
Keeps the minimum over a sliding window, allows for inserting elements to the right and for removing all elements from the left until a certain index. All operations amortized \bigo{1}.
\lstinputlisting{source/minqueue.cpp}

\subsection{Mo queries}
Offline queries on a range in \bigo{(N+Q)\sqrt{N}}. Complete the add, remove and getresult functions.
\lstinputlisting{source/mo_queries.cpp}

\subsection{LineContainer - Convex Hull Trick}

\bigo{logN} time per operation \\
\verb|LineContainer cht;| \\
\verb|cht.add(k, m);| to add a line of the form $y = kx + m$ \\
\verb|cht.query(x);| to query the maximum value with coordinate $x$
\lstinputlisting{source/line_container.cpp}

\subsection{Treap}

Stores a sequence of values and allows access by index. Supports \texttt{split} and \texttt{merge} operations, and generic queries and lazy updates on ranges. Subsequence reversal and range sums are implemented as an example, change \texttt{Treap::recalc} and \texttt{Treap::push} (and fields in \texttt{Treap} if needed) to implement new operations.
The operations modify the tree, so always reassign the tree to the return value of operations: \texttt{t = insert(t, i, val)}. To perform a range query/update, use \texttt{range\_operation}.
All operations \bigo{logN}.

Usage:
\begin{lstlisting}
Treap *t = NULL;
t = insert(t, i, s[i]);

int value;
t = erase(t, pos, &value); // third argument can also be NULL

auto [l,r] = split(t, pos); // don't use t after this

t = range_operation(t, a, b, [](Treap *u) {
	u->rev ^= 1;
});

visit(t, [](Treap *u) {
	cout << u->val;
});
\end{lstlisting}
\lstinputlisting{source/treap.cpp}


\subsection{Link Cut Tree}
\bigo{\log N} per operation
\lstinputlisting{source/link_cut_tree.cpp}
